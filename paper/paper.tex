\documentclass[proceed]{article}
\usepackage{amsmath}
\usepackage{amsfonts}
\usepackage{multicol} \usepackage{fancyheadings} \usepackage{pdfpages}
\setlength{\emergencystretch}{10em}

\newtheorem{theorem}{Theorem}[section]
\newtheorem{lemma}[theorem]{Lemma}
\newtheorem{proposition}[theorem]{Proposition}
\newtheorem{corollary}[theorem]{Corollary}
 
\newenvironment{proof}[1][Proof]{\begin{trivlist}
\item[\hskip \labelsep {\bfseries #1}]}{\end{trivlist}}
\newenvironment{definition}[1][Definition]{\begin{trivlist}
\item[\hskip \labelsep {\bfseries #1}]}{\end{trivlist}}
\newenvironment{example}[1][Example]{\begin{trivlist}
\item[\hskip \labelsep {\bfseries #1}]}{\end{trivlist}}
\newenvironment{remark}[1][Remark]{\begin{trivlist}
\item[\hskip \labelsep {\bfseries #1}]}{\end{trivlist}}

\DeclareMathOperator*{\Exp}{\mathbb{E}}
\DeclareMathOperator*{\Prob}{\mathbf{P}}


\title{Qualitative Probabilistic Programming}
\author {}
\begin{document}

  \maketitle

  \begin{abstract}
    Probabilistic programs with unknown functions
  \end{abstract}

  \section{Introduction}

  Probabilistic programming languages have made it easier to specify Bayesian
  models as programs.  Once a probabilistic model is written as a probabilistic
  program, it is possible to use generic inference algorithms (such as
  Metropolis Hastings) to perform inference.
    - Church example

  In this paper, we discuss a feature of probabilistic programming languages
  that makes it easy to specify models containing unknown functions.
  Specifically, we allow programs to specify that some random functions
  are unknown.  The system will fill in these unknown functions with
  some reasonable default class of distributions based on the types and
  proceed to infer the parameters to this distribution using the
  expectation maximization algorithm.

  This feature has multiple advantages.  First, it is easier to write
  models without knowing about the class of models being used.  This should
  make probabliistic programming more accessible to non-experts.  Secondly,
  parameter inference is more efficient if specialized algorithms are used
  rather than the generic algorithms used to infer other random variables
  (such as Metropolis Hastings).

  We define an example probabilistic programming language with this feature
  (Quipp) and show how it can be used to write machine learning models
  concisely.


  - Motivation for Quipp
    - Explanation of "unknown functions"
    - Writing machine learning algorithms as probabilistic programs
    - Accessibility to non-experts
    - Comparison to existing probabilistic programming languages
      - In other languages, use random variables for parameters
      - Random variables slower because they are updated independently

  \section{Syntax}

  

  \section{Algorithm}

    To perform inference of both latent variables and parameters, we first
    translate the program to a factor graph.  Next, we run the
    expectation maximization algorithm on this graph, iterating stages of
    estimating latent variables using Metropolis Hastings and parameter
    inference using Newton's method.

    
  - EM algorithm
    - E step is Metropolis Hastings as in probabilistic programming
    - M step is parameter optimization
  - Types
    - Brief overview of exponential families
    - Gaussian
    - Categorical
    - Other exponential families could be added easily
    - Some algebraic data types
      - Tuples
      - Either
  - Conversion to factor graph
    - Different types of nodes (exponential family)
    - Different types of factors
      - Constant factors
      - Unifying values (e.g. conditioning on compound value equaling something)
      - Factors from known random functions (e.g. gaussian)
      - Factors from unknown function calls
    - Example: clustering conversion to factor graph
  - Inference in factor graph
    - Metropolis hastings
    - Proposal distribution inferred from nearby nodes

  \section{Examples}

  \section{Discussion}

  - Comments about results
  - Future directions
    - Unbounded models
    - Recursive algebraic data types


\end{document}

